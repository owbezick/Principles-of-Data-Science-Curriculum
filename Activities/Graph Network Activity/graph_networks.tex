\documentclass[]{article}
\usepackage{lmodern}
\usepackage{amssymb,amsmath}
\usepackage{ifxetex,ifluatex}
\usepackage{fixltx2e} % provides \textsubscript
\ifnum 0\ifxetex 1\fi\ifluatex 1\fi=0 % if pdftex
  \usepackage[T1]{fontenc}
  \usepackage[utf8]{inputenc}
\else % if luatex or xelatex
  \ifxetex
    \usepackage{mathspec}
  \else
    \usepackage{fontspec}
  \fi
  \defaultfontfeatures{Ligatures=TeX,Scale=MatchLowercase}
\fi
% use upquote if available, for straight quotes in verbatim environments
\IfFileExists{upquote.sty}{\usepackage{upquote}}{}
% use microtype if available
\IfFileExists{microtype.sty}{%
\usepackage{microtype}
\UseMicrotypeSet[protrusion]{basicmath} % disable protrusion for tt fonts
}{}
\usepackage[margin=1in]{geometry}
\usepackage{hyperref}
\hypersetup{unicode=true,
            pdftitle={Graph Networks Activity},
            pdfborder={0 0 0},
            breaklinks=true}
\urlstyle{same}  % don't use monospace font for urls
\usepackage{color}
\usepackage{fancyvrb}
\newcommand{\VerbBar}{|}
\newcommand{\VERB}{\Verb[commandchars=\\\{\}]}
\DefineVerbatimEnvironment{Highlighting}{Verbatim}{commandchars=\\\{\}}
% Add ',fontsize=\small' for more characters per line
\usepackage{framed}
\definecolor{shadecolor}{RGB}{248,248,248}
\newenvironment{Shaded}{\begin{snugshade}}{\end{snugshade}}
\newcommand{\KeywordTok}[1]{\textcolor[rgb]{0.13,0.29,0.53}{\textbf{#1}}}
\newcommand{\DataTypeTok}[1]{\textcolor[rgb]{0.13,0.29,0.53}{#1}}
\newcommand{\DecValTok}[1]{\textcolor[rgb]{0.00,0.00,0.81}{#1}}
\newcommand{\BaseNTok}[1]{\textcolor[rgb]{0.00,0.00,0.81}{#1}}
\newcommand{\FloatTok}[1]{\textcolor[rgb]{0.00,0.00,0.81}{#1}}
\newcommand{\ConstantTok}[1]{\textcolor[rgb]{0.00,0.00,0.00}{#1}}
\newcommand{\CharTok}[1]{\textcolor[rgb]{0.31,0.60,0.02}{#1}}
\newcommand{\SpecialCharTok}[1]{\textcolor[rgb]{0.00,0.00,0.00}{#1}}
\newcommand{\StringTok}[1]{\textcolor[rgb]{0.31,0.60,0.02}{#1}}
\newcommand{\VerbatimStringTok}[1]{\textcolor[rgb]{0.31,0.60,0.02}{#1}}
\newcommand{\SpecialStringTok}[1]{\textcolor[rgb]{0.31,0.60,0.02}{#1}}
\newcommand{\ImportTok}[1]{#1}
\newcommand{\CommentTok}[1]{\textcolor[rgb]{0.56,0.35,0.01}{\textit{#1}}}
\newcommand{\DocumentationTok}[1]{\textcolor[rgb]{0.56,0.35,0.01}{\textbf{\textit{#1}}}}
\newcommand{\AnnotationTok}[1]{\textcolor[rgb]{0.56,0.35,0.01}{\textbf{\textit{#1}}}}
\newcommand{\CommentVarTok}[1]{\textcolor[rgb]{0.56,0.35,0.01}{\textbf{\textit{#1}}}}
\newcommand{\OtherTok}[1]{\textcolor[rgb]{0.56,0.35,0.01}{#1}}
\newcommand{\FunctionTok}[1]{\textcolor[rgb]{0.00,0.00,0.00}{#1}}
\newcommand{\VariableTok}[1]{\textcolor[rgb]{0.00,0.00,0.00}{#1}}
\newcommand{\ControlFlowTok}[1]{\textcolor[rgb]{0.13,0.29,0.53}{\textbf{#1}}}
\newcommand{\OperatorTok}[1]{\textcolor[rgb]{0.81,0.36,0.00}{\textbf{#1}}}
\newcommand{\BuiltInTok}[1]{#1}
\newcommand{\ExtensionTok}[1]{#1}
\newcommand{\PreprocessorTok}[1]{\textcolor[rgb]{0.56,0.35,0.01}{\textit{#1}}}
\newcommand{\AttributeTok}[1]{\textcolor[rgb]{0.77,0.63,0.00}{#1}}
\newcommand{\RegionMarkerTok}[1]{#1}
\newcommand{\InformationTok}[1]{\textcolor[rgb]{0.56,0.35,0.01}{\textbf{\textit{#1}}}}
\newcommand{\WarningTok}[1]{\textcolor[rgb]{0.56,0.35,0.01}{\textbf{\textit{#1}}}}
\newcommand{\AlertTok}[1]{\textcolor[rgb]{0.94,0.16,0.16}{#1}}
\newcommand{\ErrorTok}[1]{\textcolor[rgb]{0.64,0.00,0.00}{\textbf{#1}}}
\newcommand{\NormalTok}[1]{#1}
\usepackage{longtable,booktabs}
\usepackage{graphicx,grffile}
\makeatletter
\def\maxwidth{\ifdim\Gin@nat@width>\linewidth\linewidth\else\Gin@nat@width\fi}
\def\maxheight{\ifdim\Gin@nat@height>\textheight\textheight\else\Gin@nat@height\fi}
\makeatother
% Scale images if necessary, so that they will not overflow the page
% margins by default, and it is still possible to overwrite the defaults
% using explicit options in \includegraphics[width, height, ...]{}
\setkeys{Gin}{width=\maxwidth,height=\maxheight,keepaspectratio}
\IfFileExists{parskip.sty}{%
\usepackage{parskip}
}{% else
\setlength{\parindent}{0pt}
\setlength{\parskip}{6pt plus 2pt minus 1pt}
}
\setlength{\emergencystretch}{3em}  % prevent overfull lines
\providecommand{\tightlist}{%
  \setlength{\itemsep}{0pt}\setlength{\parskip}{0pt}}
\setcounter{secnumdepth}{0}
% Redefines (sub)paragraphs to behave more like sections
\ifx\paragraph\undefined\else
\let\oldparagraph\paragraph
\renewcommand{\paragraph}[1]{\oldparagraph{#1}\mbox{}}
\fi
\ifx\subparagraph\undefined\else
\let\oldsubparagraph\subparagraph
\renewcommand{\subparagraph}[1]{\oldsubparagraph{#1}\mbox{}}
\fi

%%% Use protect on footnotes to avoid problems with footnotes in titles
\let\rmarkdownfootnote\footnote%
\def\footnote{\protect\rmarkdownfootnote}

%%% Change title format to be more compact
\usepackage{titling}

% Create subtitle command for use in maketitle
\newcommand{\subtitle}[1]{
  \posttitle{
    \begin{center}\large#1\end{center}
    }
}

\setlength{\droptitle}{-2em}

  \title{Graph Networks Activity}
    \pretitle{\vspace{\droptitle}\centering\huge}
  \posttitle{\par}
    \author{}
    \preauthor{}\postauthor{}
    \date{}
    \predate{}\postdate{}
  

\begin{document}
\maketitle

\subsubsection{Libraries Used}\label{libraries-used}

\begin{Shaded}
\begin{Highlighting}[]
\KeywordTok{library}\NormalTok{(igraph)}
\KeywordTok{library}\NormalTok{(tidyverse)}
\KeywordTok{library}\NormalTok{(threejs)}
\KeywordTok{library}\NormalTok{(readr)}
\KeywordTok{library}\NormalTok{(knitr)}
\end{Highlighting}
\end{Shaded}

\subsubsection{The Data}\label{the-data}

We have provided both a custom data set and an .RDS file with a list
outlining various cases of dialogue from the tv show `Friends',
outlining instances of speech, the speaker, and the recipient of the
speech.

\begin{Shaded}
\begin{Highlighting}[]
\NormalTok{node_list <-}\StringTok{ }\KeywordTok{tibble}\NormalTok{(}\DataTypeTok{id =} \DecValTok{0}\OperatorTok{:}\DecValTok{5}\NormalTok{)}
\NormalTok{edge_list <-}\StringTok{ }\KeywordTok{tibble}\NormalTok{(}\DataTypeTok{Root =} \KeywordTok{c}\NormalTok{(}\DecValTok{0}\NormalTok{, }\DecValTok{0}\NormalTok{, }\DecValTok{0}\NormalTok{, }\DecValTok{1}\NormalTok{, }\DecValTok{1}\NormalTok{, }\DecValTok{2}\NormalTok{, }\DecValTok{2}\NormalTok{, }\DecValTok{2}\NormalTok{, }\DecValTok{3}\NormalTok{, }\DecValTok{3}\NormalTok{, }\DecValTok{4}\NormalTok{, }\DecValTok{4}\NormalTok{, }\DecValTok{5}\NormalTok{, }\DecValTok{5}\NormalTok{, }\DecValTok{5}\NormalTok{),}
                    \DataTypeTok{Destination =} \KeywordTok{c}\NormalTok{(}\DecValTok{1}\NormalTok{, }\DecValTok{2}\NormalTok{, }\DecValTok{4}\NormalTok{, }\DecValTok{2}\NormalTok{, }\DecValTok{5}\NormalTok{, }\DecValTok{0}\NormalTok{, }\DecValTok{1}\NormalTok{, }\DecValTok{5}\NormalTok{, }\DecValTok{2}\NormalTok{, }\DecValTok{4}\NormalTok{, }\DecValTok{2}\NormalTok{, }\DecValTok{3}\NormalTok{, }\DecValTok{1}\NormalTok{, }\DecValTok{2}\NormalTok{, }\DecValTok{4}\NormalTok{))}
\end{Highlighting}
\end{Shaded}

\begin{Shaded}
\begin{Highlighting}[]
\NormalTok{friends_edgelist <-}\StringTok{ }\KeywordTok{readRDS}\NormalTok{(}\StringTok{"edgelist.RDS"}\NormalTok{) }
\end{Highlighting}
\end{Shaded}

Then, let's view the first 10 rows of each data set:

\begin{Shaded}
\begin{Highlighting}[]
\KeywordTok{kable}\NormalTok{(friends_edgelist }\OperatorTok\StringTok{ }\KeywordTok{head}\NormalTok{(}\DecValTok{10}\NormalTok{))}
\end{Highlighting}
\end{Shaded}

\begin{longtable}[]{@{}llr@{}}
\toprule
from & to & weight\tabularnewline
\midrule
\endhead
a Casino Boss & a Tourist & 1\tabularnewline
a Casino Boss & Chandler & 1\tabularnewline
a Casino Boss & Joey & 1\tabularnewline
a Casino Boss & Phoebe & 1\tabularnewline
a Casino Boss & Rachel & 1\tabularnewline
a Crew Member & Alex & 1\tabularnewline
a Crew Member & Chandler & 1\tabularnewline
a Crew Member & Joey & 1\tabularnewline
a Crew Member & The Director & 1\tabularnewline
a Disembodied Voice & Phoebe & 1\tabularnewline
\bottomrule
\end{longtable}

\begin{Shaded}
\begin{Highlighting}[]
\KeywordTok{kable}\NormalTok{(edge_list }\OperatorTok\StringTok{ }\KeywordTok{head}\NormalTok{(}\DecValTok{10}\NormalTok{))}
\end{Highlighting}
\end{Shaded}

\begin{longtable}[]{@{}rr@{}}
\toprule
Root & Destination\tabularnewline
\midrule
\endhead
0 & 1\tabularnewline
0 & 2\tabularnewline
0 & 4\tabularnewline
1 & 2\tabularnewline
1 & 5\tabularnewline
2 & 0\tabularnewline
2 & 1\tabularnewline
2 & 5\tabularnewline
3 & 2\tabularnewline
3 & 4\tabularnewline
\bottomrule
\end{longtable}

Since the Friends network is quite a bit larger than our custom data
set, we'll use the filter command to filter out any occurances that
don't involve the show's main characters.

\begin{Shaded}
\begin{Highlighting}[]
\NormalTok{friends <-}\StringTok{ }\KeywordTok{c}\NormalTok{(}\StringTok{"Phoebe"}\NormalTok{, }\StringTok{"Monica"}\NormalTok{, }\StringTok{"Rachel"}\NormalTok{, }\StringTok{"Joey"}\NormalTok{, }\StringTok{"Ross"}\NormalTok{, }\StringTok{"Chandler"}\NormalTok{) }
\NormalTok{edgelist_without <-}\StringTok{ }\NormalTok{friends_edgelist }\OperatorTok\StringTok{ }
\StringTok{  }\NormalTok{dplyr}\OperatorTok{::}\KeywordTok{filter}\NormalTok{(}\OperatorTok{!}\NormalTok{(from }\OperatorTok\StringTok{ }\NormalTok{friends }\OperatorTok{&}\StringTok{ }\NormalTok{to }\OperatorTok\StringTok{ }\NormalTok{friends))}
\end{Highlighting}
\end{Shaded}

Lastly, to interact with the data most efficiently, we'll format each
dataset as a graph network; notice, each employs a different method of
doing so but arrives at the same result.

Custom data:

\begin{Shaded}
\begin{Highlighting}[]
\NormalTok{graph <-}\StringTok{ }\KeywordTok{graph.data.frame}\NormalTok{(edge_list, }\DataTypeTok{directed =} \OtherTok{TRUE}\NormalTok{)}
\end{Highlighting}
\end{Shaded}

Friends data:

\begin{Shaded}
\begin{Highlighting}[]
\NormalTok{edgelist_matrix <-}\StringTok{ }\KeywordTok{as.matrix}\NormalTok{(edgelist_without[ ,}\KeywordTok{c}\NormalTok{(}\StringTok{"from"}\NormalTok{, }\StringTok{"to"}\NormalTok{)]) }
\NormalTok{friends_graph <-}\StringTok{ }\NormalTok{igraph}\OperatorTok{::}\KeywordTok{graph_from_edgelist}\NormalTok{(edgelist_matrix, }\DataTypeTok{directed =} \OtherTok{FALSE}\NormalTok{) }\OperatorTok\StringTok{ }
\StringTok{  }\NormalTok{igraph}\OperatorTok{::}\KeywordTok{set.edge.attribute}\NormalTok{(}\StringTok{"weight"}\NormalTok{, }\DataTypeTok{value =}\NormalTok{ edgelist_without}\OperatorTok{$}\NormalTok{weight)}
\end{Highlighting}
\end{Shaded}

\subsubsection{Elements of a Graph
Network}\label{elements-of-a-graph-network}

There are two main characteristics of a graph netowrk: the nodes, which
are the individual items that form the netowrk, and the edges, which
signify the connections between nodes. You can use basic commands to see
the number of edges and vertices involved in any graph network as well
as other characteristics such as the diameter, which is the largest
number of connections between any two nodes.

Analysis of the custom data:

\begin{Shaded}
\begin{Highlighting}[]
\KeywordTok{E}\NormalTok{(graph) }\CommentTok{# Indicates 15 edges}
\end{Highlighting}
\end{Shaded}

\begin{verbatim}
## + 15/15 edges from 8f816d2 (vertex names):
##  [1] 0->1 0->2 0->4 1->2 1->5 2->0 2->1 2->5 3->2 3->4 4->2 4->3 5->1 5->2
## [15] 5->4
\end{verbatim}

\begin{Shaded}
\begin{Highlighting}[]
\KeywordTok{V}\NormalTok{(graph) }\CommentTok{# Indicates 6 nodes}
\end{Highlighting}
\end{Shaded}

\begin{verbatim}
## + 6/6 vertices, named, from 8f816d2:
## [1] 0 1 2 3 4 5
\end{verbatim}

\begin{Shaded}
\begin{Highlighting}[]
\KeywordTok{get_diameter}\NormalTok{(graph) }\CommentTok{# Indicates a diameter of four}
\end{Highlighting}
\end{Shaded}

\begin{verbatim}
## + 4/6 vertices, named, from 8f816d2:
## [1] 1 5 4 3
\end{verbatim}

Analysis of the Friends data:

In addition to checking these basic network characteristics, certain
commands tell you data about specific points.

For example, the degree command can be used to tell us that Joey
interacted with 348 characters:

\begin{Shaded}
\begin{Highlighting}[]
\KeywordTok{degree}\NormalTok{(friends_graph, }\DataTypeTok{v =} \StringTok{'Joey'}\NormalTok{)}
\end{Highlighting}
\end{Shaded}

\begin{verbatim}
## Joey 
##  348
\end{verbatim}

You can also use this data to create visualizations of characters'
social patterns. For example, the histogram below shows the number of
times that Chandler talked to each of his ``friends'', which, in this
case, signifies anyone that he spoke to more than fifteen times.

\begin{Shaded}
\begin{Highlighting}[]
\NormalTok{chandlers_friends <-}\StringTok{ }\KeywordTok{filter}\NormalTok{(friends_edgelist, from }\OperatorTok{==}\StringTok{ 'Chandler'}\NormalTok{) }\OperatorTok
\StringTok{  }\KeywordTok{filter}\NormalTok{(weight }\OperatorTok{>}\StringTok{ }\DecValTok{15}\NormalTok{) }\OperatorTok\StringTok{ }\KeywordTok{arrange}\NormalTok{(}\KeywordTok{desc}\NormalTok{(weight))}

\KeywordTok{ggplot}\NormalTok{(chandlers_friends, }\KeywordTok{aes}\NormalTok{(}\DataTypeTok{x =}\NormalTok{ to, }\DataTypeTok{y =}\NormalTok{ weight )) }\OperatorTok{+}\StringTok{ }\KeywordTok{geom_bar}\NormalTok{(}\DataTypeTok{stat =} \StringTok{"identity"}\NormalTok{, }\DataTypeTok{fill =} \StringTok{"Blue"}\NormalTok{) }\OperatorTok{+}
\StringTok{  }\KeywordTok{ggtitle}\NormalTok{(}\StringTok{"Chandler's Friends"}\NormalTok{) }\OperatorTok{+}\StringTok{ }\KeywordTok{xlab}\NormalTok{(}\StringTok{"Friend"}\NormalTok{) }\OperatorTok{+}\StringTok{ }\KeywordTok{ylab}\NormalTok{(}\StringTok{"Times Spoken to"}\NormalTok{) }\OperatorTok{+}\StringTok{ }
\StringTok{  }\KeywordTok{scale_fill_brewer}\NormalTok{(}\DataTypeTok{palette =} \StringTok{"Blues"}\NormalTok{)}
\end{Highlighting}
\end{Shaded}

\includegraphics{graph_networks_files/figure-latex/unnamed-chunk-11-1.pdf}

\subsubsection{Visualizing the Network}\label{visualizing-the-network}

Now that the basic elements of a graph network are understood, it's time
to visualize them. One way to do this is by making an ego graph, which
centers around a single node, the ego.

\begin{Shaded}
\begin{Highlighting}[]
\NormalTok{graph_ego <-}\StringTok{ }\KeywordTok{make_ego_graph}\NormalTok{(graph, }\DecValTok{3}\NormalTok{, }\StringTok{'0'}\NormalTok{, }\DataTypeTok{mode =} \KeywordTok{c}\NormalTok{(}\StringTok{'all'}\NormalTok{))[[}\DecValTok{1}\NormalTok{]]}
\NormalTok{fr_graph_ego <-}\StringTok{ }\KeywordTok{make_ego_graph}\NormalTok{(friends_graph, }\DecValTok{1}\NormalTok{, }\StringTok{'Joey'}\NormalTok{, }\DataTypeTok{mode =} \KeywordTok{c}\NormalTok{(}\StringTok{'all'}\NormalTok{))[[}\DecValTok{1}\NormalTok{]]}
\KeywordTok{plot}\NormalTok{(graph_ego)}
\end{Highlighting}
\end{Shaded}

\includegraphics{graph_networks_files/figure-latex/unnamed-chunk-12-1.pdf}

\begin{Shaded}
\begin{Highlighting}[]
\KeywordTok{plot}\NormalTok{(fr_graph_ego)}
\end{Highlighting}
\end{Shaded}

\includegraphics{graph_networks_files/figure-latex/unnamed-chunk-12-2.pdf}

As you can see, this still produces results that are not particularly
helpful, especially for the monstrous Friends netowork. At the very
least, we can add some colors to the homemade data that indicate each
node's proximity to the ego, number zero (the Friends network will take
a little more).

\begin{Shaded}
\begin{Highlighting}[]
\NormalTok{colors <-}\StringTok{ }\KeywordTok{c}\NormalTok{(}\StringTok{"red"}\NormalTok{, }\StringTok{"orange"}\NormalTok{,}\StringTok{"dodgerblue"}\NormalTok{, }\StringTok{"cyan"}\NormalTok{)}
\NormalTok{dists <-}\StringTok{ }\KeywordTok{distances}\NormalTok{(graph_ego, }\DataTypeTok{v =} \StringTok{'0'}\NormalTok{)}
\KeywordTok{V}\NormalTok{(graph_ego)}\OperatorTok{$}\NormalTok{color <-}\StringTok{ }\NormalTok{colors[dists}\OperatorTok{+}\DecValTok{1}\NormalTok{]}
\KeywordTok{plot}\NormalTok{(graph_ego, }\DataTypeTok{edge.arrow.size =}\NormalTok{ .}\DecValTok{05}\NormalTok{)}
\end{Highlighting}
\end{Shaded}

\includegraphics{graph_networks_files/figure-latex/unnamed-chunk-13-1.pdf}

For the Friends graph, we'll use the louvain\_cluster command to
organize the network into distinct communities, and assign those
communities to the graph object to encourage additional organization.
Then, we'll create subgraphs out of each community so they can be
analyzed independently and sorted using dplyr commands for analyzing the
nodes with the most betweenness, or involvement within the network. The
resulting list will show each community's member count and its most
central member.

\begin{Shaded}
\begin{Highlighting}[]
\CommentTok{# run louvain with edge weights }
\NormalTok{louvain_partition <-}\StringTok{ }\NormalTok{igraph}\OperatorTok{::}\KeywordTok{cluster_louvain}\NormalTok{(friends_graph, }\DataTypeTok{weights =} \KeywordTok{E}\NormalTok{(friends_graph)}\OperatorTok{$}\NormalTok{weight) }
\CommentTok{# assign communities to graph }
\NormalTok{friends_graph}\OperatorTok{$}\NormalTok{community <-}\StringTok{ }\NormalTok{louvain_partition}\OperatorTok{$}\NormalTok{membership }
\CommentTok{# see how many communities there are }
\KeywordTok{unique}\NormalTok{(friends_graph}\OperatorTok{$}\NormalTok{community) }
\end{Highlighting}
\end{Shaded}

\begin{verbatim}
## [1] 4 6 1 7 3 8 5 2
\end{verbatim}

\begin{Shaded}
\begin{Highlighting}[]
\NormalTok{communities <-}\StringTok{ }\KeywordTok{data.frame}\NormalTok{() }
\ControlFlowTok{for}\NormalTok{ (i }\ControlFlowTok{in} \KeywordTok{unique}\NormalTok{(friends_graph}\OperatorTok{$}\NormalTok{community)) \{ }
  \CommentTok{# create subgraphs for each community }
\NormalTok{  subgraph <-}\StringTok{ }\KeywordTok{induced_subgraph}\NormalTok{(friends_graph, }\DataTypeTok{v =} \KeywordTok{which}\NormalTok{(friends_graph}\OperatorTok{$}\NormalTok{community }\OperatorTok{==}\StringTok{ }\NormalTok{i)) }
  \CommentTok{# get size of each subgraph }
\NormalTok{  size <-}\StringTok{ }\NormalTok{igraph}\OperatorTok{::}\KeywordTok{gorder}\NormalTok{(subgraph) }
  \CommentTok{# get betweenness centrality }
\NormalTok{  btwn <-}\StringTok{ }\NormalTok{igraph}\OperatorTok{::}\KeywordTok{betweenness}\NormalTok{(subgraph) }
\NormalTok{  communities <-}\StringTok{ }\NormalTok{communities }\OperatorTok\StringTok{ }
\StringTok{    }\NormalTok{dplyr}\OperatorTok{::}\KeywordTok{bind_rows}\NormalTok{(}\KeywordTok{data.frame}\NormalTok{(}
      \DataTypeTok{community =}\NormalTok{ i, }
      \DataTypeTok{n_characters =}\NormalTok{ size, }
      \DataTypeTok{most_important =} \KeywordTok{names}\NormalTok{(}\KeywordTok{which}\NormalTok{(btwn }\OperatorTok{==}\StringTok{ }\KeywordTok{max}\NormalTok{(btwn))) }
\NormalTok{    ) }
\NormalTok{    ) }
\NormalTok{\} }
\NormalTok{knitr}\OperatorTok{::}\KeywordTok{kable}\NormalTok{(}
\NormalTok{  communities }\OperatorTok\StringTok{ }
\StringTok{    }\NormalTok{dplyr}\OperatorTok{::}\KeywordTok{select}\NormalTok{(community, n_characters, most_important))}
\end{Highlighting}
\end{Shaded}

\begin{longtable}[]{@{}rrl@{}}
\toprule
community & n\_characters & most\_important\tabularnewline
\midrule
\endhead
4 & 128 & Chandler\tabularnewline
6 & 109 & Joey\tabularnewline
1 & 105 & Phoebe\tabularnewline
7 & 93 & Rachel\tabularnewline
3 & 105 & Monica\tabularnewline
8 & 94 & Ross\tabularnewline
5 & 8 & Intercom\tabularnewline
2 & 8 & Bandleader\tabularnewline
2 & 8 & Dennis Phillips\tabularnewline
\bottomrule
\end{longtable}

Since three of these groups are notably smaller and led by insignificant
characters, we'll create subgraphs of each group and filter them out by
group size to ensure that only the groups lead by main characters are
included.

\begin{Shaded}
\begin{Highlighting}[]
\NormalTok{top_five <-}\StringTok{ }\KeywordTok{data.frame}\NormalTok{() }
\ControlFlowTok{for}\NormalTok{ (i }\ControlFlowTok{in} \KeywordTok{unique}\NormalTok{(friends_graph}\OperatorTok{$}\NormalTok{community)) \{ }
  \CommentTok{# create subgraphs for each community }
\NormalTok{  subgraph <-}\StringTok{ }\KeywordTok{induced_subgraph}\NormalTok{(friends_graph, }\DataTypeTok{v =} \KeywordTok{which}\NormalTok{(friends_graph}\OperatorTok{$}\NormalTok{community }\OperatorTok{==}\StringTok{ }\NormalTok{i)) }
  \CommentTok{# for larger communities }
  \ControlFlowTok{if}\NormalTok{ (igraph}\OperatorTok{::}\KeywordTok{gorder}\NormalTok{(subgraph) }\OperatorTok{>}\StringTok{ }\DecValTok{20}\NormalTok{) \{ }
    \CommentTok{# get degree }
\NormalTok{    degree <-}\StringTok{ }\NormalTok{igraph}\OperatorTok{::}\KeywordTok{degree}\NormalTok{(subgraph) }
\NormalTok{  \} }
\NormalTok{\}}
\end{Highlighting}
\end{Shaded}

FINALLY, it is time to visualize the network in a way that is visually
pleasing and reasonably organized. After adding some colors to represent
a node's community, scaling the nodes by degree, and adding colored
edges to show the root of the edge (or speaker of the dialogue), it's
time to plot the graphs. Layout one, the spherical layout, is more
visually pleasing but layout two is clearly preferable as it clusters
groups together and bases node placement on community involvement and
betweenness.

\begin{Shaded}
\begin{Highlighting}[]
\CommentTok{# Scaling by degree and coloring by community}
\KeywordTok{V}\NormalTok{(friends_graph)}\OperatorTok{$}\NormalTok{size <-}\StringTok{ }\DecValTok{3} 
\KeywordTok{V}\NormalTok{(friends_graph)}\OperatorTok{$}\NormalTok{frame.color <-}\StringTok{ "white"} 
\KeywordTok{V}\NormalTok{(friends_graph)}\OperatorTok{$}\NormalTok{color <-}\StringTok{ }\NormalTok{friends_graph}\OperatorTok{$}\NormalTok{community }
\KeywordTok{V}\NormalTok{(friends_graph)}\OperatorTok{$}\NormalTok{label <-}\StringTok{ }\KeywordTok{V}\NormalTok{(friends_graph)}\OperatorTok{$}\NormalTok{name }
\KeywordTok{V}\NormalTok{(friends_graph)}\OperatorTok{$}\NormalTok{label.cex <-}\StringTok{ }\FloatTok{1.5} 
\CommentTok{# Coloring by speaker}
\NormalTok{edge.start <-}\StringTok{ }\KeywordTok{ends}\NormalTok{(friends_graph, }\DataTypeTok{es =} \KeywordTok{E}\NormalTok{(friends_graph), }\DataTypeTok{names =}\NormalTok{ F)[,}\DecValTok{1}\NormalTok{] }
\KeywordTok{E}\NormalTok{(friends_graph)}\OperatorTok{$}\NormalTok{color <-}\StringTok{ }\KeywordTok{V}\NormalTok{(friends_graph)}\OperatorTok{$}\NormalTok{color[edge.start] }
\KeywordTok{E}\NormalTok{(friends_graph)}\OperatorTok{$}\NormalTok{arrow.mode <-}\StringTok{ }\DecValTok{0} \CommentTok{# only label central characters }
\NormalTok{v_labels <-}\StringTok{ }\KeywordTok{which}\NormalTok{(}\KeywordTok{V}\NormalTok{(friends_graph)}\OperatorTok{$}\NormalTok{name }\OperatorTok\StringTok{ }\NormalTok{friends) }
\ControlFlowTok{for}\NormalTok{ (i }\ControlFlowTok{in} \DecValTok{1}\OperatorTok{:}\KeywordTok{length}\NormalTok{(}\KeywordTok{V}\NormalTok{(friends_graph))) \{ }
  \ControlFlowTok{if}\NormalTok{ (}\OperatorTok{!}\NormalTok{(i }\OperatorTok\StringTok{ }\NormalTok{v_labels)) \{ }\KeywordTok{V}\NormalTok{(friends_graph)}\OperatorTok{$}\NormalTok{label[i] <-}\StringTok{ ""}\NormalTok{ \} }
\NormalTok{\}}

\NormalTok{l1 <-}\StringTok{ }\KeywordTok{layout_on_sphere}\NormalTok{(friends_graph)}
\KeywordTok{plot}\NormalTok{(friends_graph, }\DataTypeTok{rescale =}\NormalTok{ T, }\DataTypeTok{layout =}\NormalTok{ l1, }\DataTypeTok{main =} \StringTok{"'Friends' Network - All Seasons"}\NormalTok{)}
\end{Highlighting}
\end{Shaded}

\includegraphics{graph_networks_files/figure-latex/unnamed-chunk-16-1.pdf}

\begin{Shaded}
\begin{Highlighting}[]
\NormalTok{l2 <-}\StringTok{ }\KeywordTok{layout_with_mds}\NormalTok{(friends_graph) }
\KeywordTok{plot}\NormalTok{(friends_graph, }\DataTypeTok{rescale =}\NormalTok{ T, }\DataTypeTok{layout =}\NormalTok{ l2, }\DataTypeTok{main =} \StringTok{"'Friends' Network - All Seasons"}\NormalTok{)}
\end{Highlighting}
\end{Shaded}

\includegraphics{graph_networks_files/figure-latex/unnamed-chunk-16-2.pdf}

\subsubsection{Acknowledgements}\label{acknowledgements}

Special thanks to Keith McNulty's article ``Community Detection in R
Using Communitues of Friends Characters'' for adding some popular
culture to this activity.


\end{document}
